% \iffalse meta-comment
% fithesis.dtx, consolidated stable version, encoded in utf8
%
% Copyright 1998--2015 Daniel Marek (DM), Jan Pavlovič (JP), 
% Vít Novotný (VN), Petr Sojka (PS)
% https://www.fi.muni.cz/tech/unix/tex/fithesis.xhtml
% Faculty of Informatics, Masaryk University
%
% This work may be distributed and/or modified under the
% conditions of the LaTeX Project Public License, either version 1.3
% of this license or (at your option) any later version.
% The latest version of this license is in
%   http://www.latex-project.org/lppl.txt
% and version 1.3 or later is part of all distributions of LaTeX
% version 2005/12/01 or later.
%
% This work has the LPPL maintenance status `maintained'.
% 
% The Current Maintainer of this work is NO LONGER Jan Pavlovič,
% Send bug reports, requests for additions and questions
% to the fithesis discussion forum at
% <https://is.muni.cz/auth/df/fithesis-sazba/>.
%
% The package will be superceded by new version named 
% fithesis3 in June 2015.
%
% This work consists of the files fithesis.dtx and fithesis.ins
% and the derived files fithesis2.cls, fithesis.cls, fit10.clo,
% fit11.clo, fit12.clo.
%
% TODO:
% - commented source, in English
% - adding reference to docbook
%
%    \begin{macrocode}
%<*driver>
\documentclass{ltxdoc}
\usepackage[utf8]{inputenc} % this file uses UTF-8
\usepackage[czech]{babel}
\usepackage{tgpagella}
\usepackage[scaled=0.86]{berasans}
\usepackage[scaled=1.03]{inconsolata}
\usepackage[resetfonts]{cmap}
\usepackage[T1]{fontenc} % use 8bit fonts
\usepackage{csquot,mflogo}
\usepackage{url}
\emergencystretch 2dd

\CodelineIndex % ltxdoc class index
\EnableCrossrefs
\GlossaryPrologue{
  \section*{Změny}
}
\IndexPrologue{
  \section*{Rejstřík}
  Číslice vysázené kurzívou odkazují na stránku, na které je příslušná
položka popsána. Podtržené číslice odkazují na řádek v kódu, na kterém
je položka definována. Číslice v základním řezu odkazují na řádky kódu,
na nichž je položka použita.
}
\DoNotIndex{\def,\gdef,\hypersetup,\ifx,\fi,\True,\False,\\else,\\end,
  \\begin,\',\\\\,\bf,\footnoterule,\footnotesize,\large,\Large,\let,
  \newpage,\newcommand,\pagenumbering,\pagestyle,\par,\parindent,
  \relax,\setcounter,\small,\v,\vfil,\vfill,\vskip,\newenvironment,
  \chapter,\cleardoublepage,\clearpage,\null,\\if@twoside,
  \includegraphics}

\begin{document}
  \RecordChanges
  \DocInput{fithesis.dtx}
  \PrintIndex
  \PrintChanges
\end{document}
%</driver>
%    \end{macrocode}
% \fi
%
%%%%%%%%%%%%%%%%%%%%%%%%%%%%%%%%%%%%%%%%%%%%%%%%%%%%%%%%%%%%%%%%%%%%%%%%%%%%%%%
%
% \changes{v0.2.18} {2015/03/04}{Fixed a non-terminated \cs{if} condition. [VN]}
% \changes{v0.2.17} {2015/03/04}{Fixed mostly documentation errors reported
%   at the new fithesis discussion forum (-ti, eco$\rightarrow$econ, 
%   implicit twocolumn, example extended (font setup), etc.). [PS]} 
% \changes{v0.2.16} {2015/01/21}{Added support for change typesetting.
%   Restructured the code to make it more amenable to literal programming.
%   Added support for \cs{CodelineIndex} typesetting. Added information about
%   the usage of \textsf{fithesis1} and \textsf{fithesis2} on the FI unix
%   machines. [VN]}
% \changes{v0.2.15} {2015/01/14}{The rest of the file converted to utf8
%   Import the url package to allow for the use of \cs{url} within the
%   documentation. Added deprecation error msg for fithesis1. [VN]}
% \changes{v0.2.14}  {2015/01/14}{Small fixes (added \cs{relax} at
%   \cs{MainMatter}), generating both fithesis.cls (obsolete, loading
%   \texttt{fithesis.cls}) and \texttt{fithesis2.cls}, minor doc edits,
%   version numbering of \texttt{.clo} fixed, switch to utf8 and ensuring that
%   \texttt{.dtx} compiles. Documentation adjusted to the status quo, added
%   link to discussion forum. [PS]}
% \changes{v0.2.13} {2015/01/09}{Removed extraneous \cs{fi} on line 1129. [VN]}
% \changes{v0.2.12a}{2008--2011}{fork fithesis2 by Mr. Filipčík and Janoušek;
%   cf. \protect\url{https://github.com/liskin/fithesis}}
% \changes{v0.2.12} {2008/07/27}{Licence change to the LPPL [JP]}
% \changes{v0.2.11} {2008/01/07}{fix missing \texttt{fi-logo.mf} [JP,PS]}
% \changes{v0.2.10} {2006/05/12}{fix EN name of Acknowledgement [JP]}
% \changes{v0.2.09}  {2006/05/08}{add EN version of University name [JP]}
% \changes{v0.2.08}  {2006/01/20}{add change of University name [JP]}
% \changes{v0.2.07}  {2005/05/10}{escape all Czech letters [JP]
%   babel is used instead of stupid package czech [JP]
%   \cs{MainMatter} should be placed after \cs{tablesofcontents} [PS]}
% \changes{v0.2.06}  {2004/12/22}{fix : behind Advisor [JP]}
% \changes{v0.2.05}  {2004/05/13}{add English abstract [JP]}
% \changes{v0.2.04}  {2004/05/13}{fix SK declaration [Peter Cerensky, JP]}
% \changes{v0.2.03}  {2004/05/13}{fix title spacing [PS, JP]}
% \changes{v0.2.02}  {2004/05/12}{fix encoding bug [JP]}
% \changes{v0.2.01}  {2004/05/11}{add subsubsection to toc [JP]}
% \changes{v0.2.00}  {2004/05/03}{add sk lang [JP, Peter Cerensky]
%   set default cls class to \textsf{rapport3} [JP]}
% \changes{v0.1g}   {2004/04/01}{change of default size (12pt$\rightarrow$11pt) [JP]}
% \changes{v0.1f}   {2004/01/24}{add documentation for hyperref [JP]}
% \changes{v0.1e}   {2004/01/07}{add Brno to MU title [JP]}
% \changes{v0.1d}   {2003/03/24}{removed def schapter from fit1*.clo [JP]}
% \changes{v0.1c}   {2003/02/21}{default values of \cs{facultyname} and
%   \\\cs{@thesissubtitle} set for backward compatibility) [PS]}
% \changes{v0.1b}   {2003/02/14}{change of default size (11pt$\rightarrow$12pt) [JP]}
% \changes{v0.1a}   {2003/02/12}{minor documentation changes (CZ only,
%   sorry) [PS]}
% \changes{v0.1}    {2003/02/11}{new release, documentation editing (CZ only,
%   sorry) [PS]}
% \changes{v0.0a}   {2002}{changes by Jan Pavlovič to allow fithesis being
%   backend of docbook based system for thesis writing}
% \changes{v0.0}    {1998}{bachelor project of Daniel Marek under
%   supervision of Petr Sojka}
%
%%%%%%%%%%%%%%%%%%%%%%%%%%%%%%%%%%%%%%%%%%%%%%%%%%%%%%%%%%%%%%%%%%%%%%%%%%%%%%%
%
% \newcommand{\bs}{\char`\\}
% \newcommand{\prikaz}[1]{\texttt{\bs #1}}
% \newcommand{\fit}{\textsf{fithesis2}}
% \newcommand{\itm}[1]{\noindent{\bf #1}} 
%
% \title{Sada maker \fit\ pro~sazbu~závěrečných~prací~MU}
% \author{Daniel Marek, Jan Pavlovič, Petr Sojka, Vít Novotný}
% \date{\today}
% \maketitle
%
% \begin{abstract}
% \noindent 
% Tento text popisuje instalaci a použití sady \LaTeX ových maker
% pro sazbu diplomové či bakalářské práce na fakultách
% Masarykovy univerzity. Uživateli umožní jednotně
% vysadit všechny potřebné povinné i nepovinné části
% stanovené v~pokynech pro vypracování závěrečných prací 
% na MU, jež pro FI najdete na 
% \url{http://www.fi.muni.cz/studies/BP_DP.xhtml}.
% Použití stylu však automaticky \emph{nezaručuje}
% typografickou správnost. Používejte jako
% pomůcku, která pravděpodobně zajistí jednotnost
% sazby jednotlivých elementů práce.
% \end{abstract}
%
% \tableofcontents
%
% \section{Instalace maker \texttt{fithesis}}
% K~samotné instalaci stylu jsou potřeba alespoň dva soubory:
% standardní instalační soubory \LaTeX u 
% \texttt{fithesis.dtx} a \texttt{fithesis.ins}.
% Protože je v~makrech používáno písmo Palatino
% (verze z~balíku TeX Gyre), logo
% Fakulty informatiky a samotná sazba diplomové a bakalářské práce je
% založena na stylu \textsf{scrreprt}, je třeba zároveň  
% instalovat i tuto podporu, pokud ji distribuce \TeX u, kterou
% používáte, ještě neobsahuje.
%
% Instalaci je možné automatizovat programem \texttt{make}
% přiloženým \texttt{Makefile}.
%
% Po spuštění instalace příkazem \texttt{tex fithesis.ins} se vygenerují
% soubory \texttt{fithesis.cls} (třída \textsf{fithesis1}), 
% \texttt{fithesis2.cls} (základní třída) a soubory 
% \texttt{fit10.clo}, \texttt{fit11.clo} a \texttt{fit12.clo} 
% (volby určující velikosti písem a mezerování). 
% Příkazy
% \begin{verbatim}
% pdflatex fithesis.dtx
% makeindex fithesis
% pdflatex fithesis.dtx
% \end{verbatim} 
% je možné přeložit dokumentaci. Další, již ne zcela aktuální,
% dokumentace ve formě bakalářské práce pana Filipčíka je v~souboru    
% \texttt{tutorial.pdf}.
%
% Na fakultních strojích se v~aktuální distribuci modulu \texttt{texlive}
% nachází pouze třída \textsf{fithesis2}. Pokud chcete použít třídu
% \textsf{fithesis1}, nahraďte jej modulem \texttt{texlive-2013} nebo starším
% pomocí příkazu \begin{verbatim}module switch texlive texlive-2013\end{verbatim}
%
% Užití stylu je na MU podporováno, návrhy a připomínky jsou vítány
% na diskusním fóru Informačního systému MU na URL\newline
% \url{https://is.muni.cz/auth/df/fithesis-sazba/}.
% Je připravována verze \texttt{fithesis3} s~dokumentací v~angličtině a
% mnohými novými možnostmi a zlepšeními v~rámci bakalářské 
% práce pana Víta Novotného.
%
% \iffalse
%    \begin{macrocode}
%<*class>
\NeedsTeXFormat{LaTeX2e}
\ProvidesClass{fithesis2}[2015/03/04 fithesis version 0.2.18 MU thesis class]

\ifx\clsclass\undefined
 \def\clsclass{rapport3}
\fi
\LoadClass[a4paper]{\clsclass}

%</class>
%    \end{macrocode}
% \fi
%
% \section{Použití třídy \fit}
% Pro použití sady maker uvedeme v~příkazu \prikaz{documentclass}
% vytvářeného dokumentu třídu (styl) \fit, která může být modifikována 
% volbami, umístěnými ve volitelném parametru tohoto příkazu. 	
% Možné volby jsou tyto:
% \begin{itemize}
% \item [--]{\bf 10pt} -- Změní základní velikost písma na 10~bodů. Při
% této volbě je počet řádek vysazené strany roven 40, průměrný počet
% znaků na řádku se pohybuje mezi 70 až 80. Nedoporučováno,
% pokud nebude při výsledném tisku tiskové zrcadlo zvětšováno z~B5
% na A4. 
% \item [--]{\bf 11pt} -- Základní velikost písma bude 11~bodů. 
% Tato volba byla ve starší verzi nastavena implicitně.
% Počet řádek vysazené strany je~40, 
% průměrný počet znaků na řádce při použití fontu Palatino
% je 65 až~70.
% \item [--]{\bf 12pt} -- Základní velikost písma se touto volbou změní na 
% 12~bodů. Počet řádek na stránce je~38, průměrný počet znaků na řádce
% je 55 až 60. Tato volba je implicitní a doporučována.
% \item [--]{\bf oneside} -- Tato volba umožní sazbu práce pouze
% jednostraně, je nastavena implicitně. Sazba je pouze 
% na stranách lichých. Tato volba je implicitní a doporučována.
% \item [--]{\bf twoside} -- Sazba práce bude oboustranná,
% rozlišují se liché a sudé strany, začátky kapitol a jiných významných
% celků jsou umístěny vždy na straně liché, tedy pravé.
% \item [--]{\bf onecolumn} -- Implicitně nastavená volba pro sazbu textu
% do jednoho sloupce na stránce. Text je zarovnaný oba okraje sloupce.
% Tato volba je implicitní a doporučována.
% \item [--]{\bf twocolumn} -- Tato volba umožní sazbu textu do dvou
% sloupců na stránku. Text je zarovnaný na oba okraje sloupce.
% \item [--]{\bf draft} -- Po nastavení této volby bude špatně zalomený
% text na koncích řádků zvýrazněn černým obdélníčkem pro snažší vizuální
% identifikaci. Dále volbu přebírají další balíky, jako je 
% \texttt{graphics}, a zde způsobí sazbu rámečků místo
% vkládání obrázků.
% \item [--]{\bf final} -- Opak volby draft. Tato volba je nastavena
% implicitně.
% \end{itemize}
% Jednotlivé volby se mohou patřičně kombinovat. Lze volit mezi velikostí
% základního písma (10pt, 11pt a 12pt), mezi sazbou jednostrannou a
% oboustrannou, sazbou jednosloupcovou a dvousloupcovou a mezi konečnou
% finální podobou a konceptem dokumentu (volby final a draft). 
% \iffalse
%    \begin{macrocode}
%<*class>

\DeclareOption{10pt}{\renewcommand\@ptsize{0}}
\DeclareOption{11pt}{\renewcommand\@ptsize{1}}
\DeclareOption{12pt}{\renewcommand\@ptsize{2}}
\DeclareOption{oneside}{\@twosidefalse \@mparswitchfalse}
\DeclareOption{twoside}{\@twosidetrue  \@mparswitchtrue}
\DeclareOption{onecolumn}{\@twolumnfalse}
\DeclareOption{twocolumn}{\@twocolumntrue}
\DeclareOption{draft}{\setlength\overfullrule{5pt}}
\DeclareOption{final}{\setlength\overfullrule{0pt}}

\ExecuteOptions{12pt,oneside,final}
\ProcessOptions

% pridat volbu, aby slo vypnout mathpazo, zapnout lmodern, atd.
\RequirePackage{tgpagella}
\RequirePackage{mathpazo}
\RequirePackage{graphicx}
% FIXME: pridat ifxetex apod.
\RequirePackage{cmap}
\RequirePackage[T1]{fontenc}

\def\Scrreprtcls{scrreprt}
\def\RapportIcls{rapport1}
\def\RapportIIIcls{rapport3}

\ifx\clsclass\RapportIcls\else
\ifx\clsclass\RapportIIIcls\else
 \newcommand*\ChapFont{\bfseries}
 \newcommand*\PageFont{\bfseries}
\fi
\fi

\setcounter{tocdepth}{4}

\input fit1\@ptsize.clo\relax

\def\ps@thesisheadings{%
\def\chaptermark##1{%
\markright{%
\ifnum\c@secnumdepth >\m@ne
\thechapter.\ %
\fi ##1}}
\let\@oddfoot\@empty
\let\@oddhead\@empty
\def\@oddhead{\vbox{\hbox to \textwidth{%
\hfil{\sc\rightmark}}\vskip 4pt\hrule}}
\if@twoside
 \def\@evenhead{\vbox{\hbox to \textwidth{%
 {\sc\rightmark}\hfil}\vskip 4pt\hrule}}
\else
 \let\@evenhead\@oddhead
\fi
\def\@oddfoot{\hfil\PageFont\thepage}
\if@twoside
 \def\@evenfoot{\PageFont\thepage\hfil}%
\else
 \let\@evenfoot\@oddfoot
\fi
\let\@mkboth\markboth
}

\renewcommand*\chapter{%
\if@twoside
 \clearpage
 \thispagestyle{empty}
 \cleardoublepage
\else
 \clearpage
\fi
\thispagestyle{plain}%
\global\@topnum\z@
\@afterindentfalse
\secdef\@chapter\@schapter}

\renewcommand*\part{%
\clearpage
\thispagestyle{empty}
\cleardoublepage
\thispagestyle{empty}%
\if@twocolumn%
 \onecolumn
 \@tempswatrue
\else
 \@tempswafalse
\fi
\hbox{}\vfil
\secdef\@part\@spart}

\def\logopath{loga/}
\font\filogo fi-logo600 at 40mm
\def\facultylogo{\logopath\@thesisfaculty-logo}
\def\universityname{Masarykova univerzita}
\def\facultyname{Fakulta informatiky}
\def\@thesissubtitle{Diplomov\'{a} pr\'{a}ce}
\def\lowecasewrapper#1{\lowercase{#1}}
\def\Fi{fi}
\def\Sci{sci}
\def\Law{law}
\def\Econ{econ}
\def\Fss{fss}
\def\Med{med}
\def\Ped{ped}
\def\Phil{phil}
\def\Fsps{fsps}
\def\True{true}

\def\Langcs{cs}
\def\Langsk{sk}
\def\Langen{en}
\def\Langcz{cz}
\def\@thesislang{cs}

\def\titlefont{\fontsize\@xxvpt{30}\selectfont}

%    \end{macrocode}
% \fi
%
% \section{Popis jednotlivých maker}
% Následující makra slouží k vložení základních údajů potřebných 
% k~vysazení titulní strany. Na titulní stranu se kromě názvu
% práce, jména studenta a roku vypracování vysadí také logo fakulty.
%
% \DescribeMacro\thesistitle
% Makro umožní vložit název práce, u dvouřádkových
% či víceřádkových názvů se standardně oddělí jednotlivé části
% příkazem $\backslash$$\backslash$ s~volitelným parametrem 
% meziřádkového prokladu. Další možností je části nadpisu 
% požadované na stejném řádku (kolokace, fráze) spojit 
% nedělitelnou předložkou, neboť v~nadpisech je zakázáno 
% dělení slov.
% 
%    \begin{macrocode}
\def\thesistitle#1{\gdef\@thesistitle{#1}}
%    \end{macrocode}
%
% \DescribeMacro\thesissubtitle
% Makro umožní vložit název typu práce, např. bakalářská práce
% diplomová práce atd.
%    \begin{macrocode}
\def\thesissubtitle#1{\gdef\@thesissubtitle{#1}}
%    \end{macrocode}
%
% \DescribeMacro\thesisstudent
% Makro umožní pomocí svého jediného parametru vložit jméno studenta.
%    \begin{macrocode}
\def\thesisstudent#1{\gdef\@thesisstudent{#1}}
%    \end{macrocode}
%
% \iffalse
%    \begin{macrocode}

\newif\if@restonecol
\def\alwayssingle{%
\@restonecolfalse\if@twocolumn\@restonecoltrue\onecolumn\fi}
\def\endalwayssingle{\if@restonecol\twocolumn\fi}

\newif\ifwoman\womanfalse
\def\@w{\ifwoman a\else\fi}

%    \end{macrocode}
% \fi
%
% \DescribeMacro\thesiswoman
% Makro umožní vložit pohlaví studenta, volby jsou: true, false 
% (nahrazuje použití přepínače \prikaz{ifwoman}).
%    \begin{macrocode}
\def\thesiswoman#1{\def\@thesiswoman{#1}
  \ifx\@thesiswoman\True\womantrue\else\womanfalse\fi}
%    \end{macrocode}
%
% \DescribeMacro\thesisfaculty
% Makro umožní stanovit pod jakou fakultou byla práce napsána. Podle toho
% se také vloží patřičné logo a název fakulty na titulní stránku.
% Jsou podporovány tyto fakulty MU:
% \begin{itemize}
% \item Fakulta informatiky -- fi\footnote{Použije se originální 
% opticky škálované logo v~jazyce~\MF{}.},  
% \item Přírodovědecká fakulta -- sci,
% \item Právnická fakulta -- law,
% \item Ekonomicko-správní fakulta -- econ,
% \item Fakulta sociálních studií -- fss,
% \item Lékařská fakulta -- med,
% \item Pedagogická fakulta -- ped,
% \item Filozofická fakulta -- phil,
% \item Fakulta sportovních studií -- fsps.
% \end{itemize}
% například: \prikaz{thesisfaculty\{fi\}}.
% Lze použít i vlastní název, pokud práce není psaná pod 
% žádnou z~výše uvedených fakult MU, pak je nutné zadat 
% i název univerzity \prikaz{thesisuniversity\{\}}, 
% jméno souboru loga fakulty (bez přípony) 
% \prikaz{thesislogo\{\}} a též do makra 
% \prikaz{thesisyear\{\}} sídlo dané univerzity 
% (pro MU toto není třeba).
%    \begin{macrocode}
\def\thesisfaculty#1{\gdef\@thesisfaculty{#1}%
\def\tmp{eco}
\ifx\@thesisfaculty\tmp
 \gdef\@thesisfaculty{econ}
\fi
\ifx\@thesisfaculty\Fi
 \ifx\@thesislang\Langen
  \def\facultyname{Faculty of Informatics}
  \def\universityname{Masaryk University}
   \else \def\facultyname{Fakulta informatiky}
  \fi
 \else \ifx\@thesisfaculty\Sci
  \ifx\@thesislang\Langen
   \def\facultyname{Faculty of Science}
   \def\universityname{Masaryk University}
  \else \def\facultyname{P\v{r}\'{i}rodov\v{e}deck\'{a} fakulta}
  \fi
  \else \ifx\@thesisfaculty\Law
   \ifx\@thesislang\Langen
    \def\facultyname{Faculty of Law}
    \def\universityname{Masaryk University}
   \else \def\facultyname{Pr\'{a}vnick\'{a} fakulta}
   \fi
  \else \ifx\@thesisfaculty\Econ
   \ifx\@thesislang\Langen
    \def\facultyname{Faculty of Economics and Administration}
    \def\universityname{Masaryk University}
   \else \def\facultyname{Ekonomicko-spr\'{a}vn\'{i} fakulta}
   \fi
  \else \ifx\@thesisfaculty\Fss
   \ifx\@thesislang\Langen
    \def\facultyname{Faculty of Social Studies}
    \def\universityname{Masaryk University}
   \else \def\facultyname{Fakulta soci\'{a}ln\'{i}ch studi\'{i}}
   \fi
  \else \ifx\@thesisfaculty\Med
   \ifx\@thesislang\Langen
    \def\facultyname{Faculty of Medicine}
    \def\universityname{Masaryk University}
   \else \def\facultyname{L\'{e}ka\v{r}sk\'{a} fakulta}
   \fi
  \else \ifx\@thesisfaculty\Ped
   \ifx\@thesislang\Langen
    \def\facultyname{Faculty of Education}
    \def\universityname{Masaryk University}
   \else \def\facultyname{Pedagogick\'{a} fakulta}
   \fi
  \else \ifx\@thesisfaculty\Phil
   \ifx\@thesislang\Langen
    \def\facultyname{Faculty of Arts}
    \def\universityname{Masaryk University}
   \else \def\facultyname{Filozofick\'{a} fakulta}
   \fi
  \else \ifx\@thesisfaculty\Fsps
   \ifx\@thesislang\Langen
    \def\facultyname{Faculty of Sports Studies}
    \def\universityname{Masaryk University}
   \else \def\facultyname{Fakulta sportovn\'{i}ch studi\'{i}}
   \fi
         \else
          \def\facultyname{\@thesisfaculty}
          \def\universityname{\@thesisuniversity}
          \def\facultylogo{\@thesislogo}
          \def\thesisplaceyear{\@thesisyear}
         \fi
        \fi
       \fi
      \fi
     \fi
    \fi
   \fi
  \fi
\fi
}
%    \end{macrocode}
% \DescribeMacro\thesisyear
% Makro umožní vložit rok vypracování práce. 
%    \begin{macrocode}
\def\thesisyear#1{\gdef\@thesisyear{#1}}
\def\thesisplaceyear{Brno, \@thesisyear}
%    \end{macrocode}
%
% \DescribeMacro\thesisadvisor
% Makro umožní vložit jméno vedoucího práce.
%    \begin{macrocode}
\def\thesisadvisor#1{\gdef\@thesisadvisor{#1}}
%    \end{macrocode}
%
% \DescribeMacro\thesisuniversity
% Makro umožní stanovit pod jakou univerzitou byla práce napsána.
% Má význam jen v~případě, že práce není psaná pod MU.
%    \begin{macrocode}
\def\thesisuniversity#1{\gdef\@thesisuniversity{#1}}
%    \end{macrocode}
%
% \DescribeMacro\thesislogo
% Makro umožní stanovit soubor (bez přípony) loga fakulty pod
% jakou byla práce napsaná.
% Má význam jen v~případě, že práce není psaná pod MU.
%    \begin{macrocode}
\def\thesislogo#1{\gdef\@thesislogo{#1}}
%    \end{macrocode}
%
% \DescribeMacro\thesislang
% Makro umožní stanovit jazyk, ve kterém je práce napsána. V~současné
% době jsou podporovány varianty cs nebo cz, sk a en. Jazyk je třeba
% stanovit před použitím příkazu \prikaz{thesisfaculty}, jinak dojde
% k~vysázení jména fakulty v~češtině.
%    \begin{macrocode}
\def\thesislang#1{\gdef\@thesislang{#1}%
  \ifx\@thesislang\Langcz\gdef\@thesislang{cs}\fi}
%    \end{macrocode}
%
% \iffalse
%    \begin{macrocode}

\def\DeclarationTextcs{%
	Prohla\v{s}uji, \v{z}e tato \expandafter\lowecasewrapper\@thesissubtitle{} 
	je m\'{y}m p\r{u}vodn\'{i}m autorsk\'{y}m
	d\'{i}lem, kter\'{e} jsem vypracoval\@w\ samostatn\v{e}. V\v{s}echny zdroje, prameny a
	literaturu, kter\'{e} jsem p\v{r}i vypracov\'{a}n\'{i} pou\v{z}\'{i}val\@w\ nebo z~nich
	\v{c}erpal\@w, v~pr\'{a}ci \v{r}\'{a}dn\v{e} cituji s~uveden\'{i}m
	\'{u}pln\'{e}ho odkazu na p\v{r}\'{i}slu\v{s}n\'{y} zdroj.}
\def\DeclarationTextsk{%
	Prehlasujem, \v{z}e t\'{a}to \expandafter\lowecasewrapper\@thesissubtitle{} 
	je moj\'{i}m p\^{o}vodn\'{y}m autorsk\'{y}m
	dielom, ktor\'{e} som vypracoval\@w\ samostatne. V\v{s}etky zdroje, pramene a
	literat\'{u}ru, ktor\'{e} som pri vypracovan\'{i} pou\v{z}\'{i}val\@w\ alebo z~nich
	\v{c}erpal\@w, v~pr\'{a}ci riadne citujem s~uveden\'{i}m
	\'{u}pln\'{e}ho odkazu na pr\'{i}slu\v{s}n\'{y} zdroj.}
\def\DeclarationTexten{%
	Hereby I declare, that this paper is my original authorial work, 
	which I have worked out by my own. All sources, references and
	literature used or excerpted during elaboration of this work 
	are properly cited and listed in complete reference to the due source.}

\def\DeclarationTitlecs{%
	Prohl\'{a}\v{s}en\'{i}}

\def\DeclarationTitlesk{%
	Prehl\'{a}senie}

\def\DeclarationTitleen{%
	Declaration}

\def\ThanksTitlecs{%
	Pod\v{e}kov\'{a}n\'{i}} 

\def\ThanksTitlesk{%
	Po\v{d}akovanie}

\def\ThanksTitleen{%
	Acknowledgement} 

\def\AbstractTitlecs{%
	Shrnut\'{i}}

\def\AbstractTitlesk{%
	Zhrnutie}

\def\AbstractTitleen{%
	Abstract}

\def\KeyWordsTitlecs{%
	Kl\'{i}\v{c}ov\'{a} slova}

\def\KeyWordsTitlesk{%
	K\v{l}\'{u}\v{c}ov\'{e} slov\'{a}}

\def\KeyWordsTitleen{%
	Keywords}

\def\AdvisorTitlecs{%
	Vedouc\'{i} pr\'{a}ce:
}

\def\AdvisorTitlesk{%
	Ved\'{u}ci pr\'{a}ce:
}

\def\AdvisorTitleen{%
	Advisor:
}

\def\DeclarationText{%
	\ifx\@thesislang\Langcs
	 \DeclarationTextcs
	 \else \ifx\@thesislang\Langsk
	  \DeclarationTextsk
	  \else \ifx\@thesislang\Langen
	   \DeclarationTexten
	   \else \DeclarationTextcs
	  \fi
	 \fi
	\fi
	\vskip 2cm
	\hfill\@thesisstudent
}

\def\AdvisorName{\par\vfill{
\ifx\@thesislang\Langcs
 \bf \AdvisorTitlecs
 \else \ifx\@thesislang\Langsk
  \bf \AdvisorTitlesk
  \else \ifx\@thesislang\Langen
   \bf \AdvisorTitleen
   \else \bf \AdvisorTitlecs
  \fi
 \fi
\fi} \@thesisadvisor}

%    \end{macrocode}
% \fi
%
% \DescribeMacro{\FrontMatter}
% Toto makro se vloží na začátek dokumentu (nejlépe za příkaz
% \prikaz{begin\{document\}}). 
% První strany dokumentu obsahujících prohlášení, abstrakt a klíčová
% slova se nastaví na římské číslování. U~dalších stran včetně
% obsahu a následujících kapitol se pomocí makra \prikaz{MainMatter} 
% nastaví arabské číslování.
%    \begin{macrocode}
\def\FrontMatter{%
  \pagestyle{plain}
  \parindent 1.5em
  \setcounter{page}{1}
  \pagenumbering{roman}}
%    \end{macrocode}
%
% \DescribeMacro\ThesisTitlePage
% Titulní strana práce se vysadí příkazem 
% \prikaz{ThesisTitlePage} a využije předem zadané údaje
% názvu práce a jména studenta a roku vypracování.
%    \begin{macrocode}
  \newcommand{\ThesisTitlePage}{%
  \begin{alwayssingle}
  \thispagestyle{empty}
  \begin{center}
  {\sc \universityname\\ \facultyname}
  \vskip 1em

  \ifx\@thesisfaculty\Fi
   {\filogo SL}\\[0.4in]
  \else
   \includegraphics[width=40mm]{\facultylogo}\\[0.4in]
  \fi

  \let\footnotesize\small
  \let\footnoterule\relax{}
  {\titlefont\bf\@thesistitle\par\vfil}\vskip 0.8in
  {\sc \@thesissubtitle}\\[0.3in]
  {\Large\bf\@thesisstudent}
  \par\vfill
  {\large \thesisplaceyear}
  \end{center}
  \end{alwayssingle}
  \newpage}
%    \end{macrocode}
%
% \subsubsection*{Povinné části diplomové práce}
% Následující makra jsou potřebná k~vysazení povinných částí diplomové
% práce. Jsou jimi \textit{prohlášení o samostatném vypracování},
% \textit{shrnutí diplomové práce} a \textit{klíčová slova}. 
% Nepovinnou částí je \textit{poděkování\/}. 
% Pro všechny tyto celky je vždy definováno prostředí,
% které zajistí kromě vysazení každé části na samostatnou stranu
% například také jednotné styly nadpisů. Poslední povinnou 
% částí je \textit{seznam literatury}, ten se, stejně jako 
% \textit{obsah diplomové práce} již sází pomocí 
% standardních \LaTeX ových příkazů. 
%
% \DescribeMacro\ThesisDeclaration
% Prostředí \texttt{ThesisDeclaration} vysadí stránku 
% s~prohlášením o samostatném vypracování
% diplomové práce. Text tohoto prohlášení může uživatel předefinovat
% pomocí makra \prikaz{DeclarationText}. Implicitně sázený text je
% následovný: 
% \begin{quote}{\it
% Prohlašuji, že tato diplomová práce je mým původním autorským
% dílem, které jsem vypracoval samostatně. Všechny zdroje, prameny a
% literaturu, které jsem při vypracování používal nebo z~nich
% čerpal, v~práci řádně cituji s~uvedením úplného odkazu na příslušný
% zdroj.}
% \end{quote}
% Dále se vloží makro \prikaz{AdvisorName}, které vysází 
% údaje o vedoucím práce.
%    \begin{macrocode}
\newenvironment{ThesisDeclaration}{%
  \begin{alwayssingle}
  \ifx\@thesislang\Langcs
   \chapter*{\DeclarationTitlecs}
   \else \ifx\@thesislang\Langsk
    \chapter*{\DeclarationTitlesk}
    \else \ifx\@thesislang\Langen
     \chapter*{\DeclarationTitleen}
     \else \chapter*{\DeclarationTitlecs}
    \fi
   \fi
  \fi}
  {\par\vfil
  \end{alwayssingle}
  \newpage}
%    \end{macrocode}
%
% \DescribeMacro\ThesisThanks
% Toto prostředí umožní vysadit \textit{poděkování\/}.
%    \begin{macrocode}
\newenvironment{ThesisThanks}{%
  \begin{alwayssingle}
  \ifx\@thesislang\Langcs
   \chapter*{\ThanksTitlecs}
   \else \ifx\@thesislang\Langsk
    \chapter*{\ThanksTitlesk}
    \else \ifx\@thesislang\Langen
     \chapter*{\ThanksTitleen}
     \else \chapter*{\ThanksTitlecs}
    \fi
   \fi
  \fi}
  {\par\vfill
  \end{alwayssingle}
  \newpage}
%    \end{macrocode}
%
% \DescribeMacro\ThesisAbstract
% \textit{Shrnutí\/} diplomové práce je možno vysadit pomocí 
% prostředí \texttt{ThesisAbstract}. Shrnutí by mělo 
% zabírat prostor nejvýše jedné strany. 
%    \begin{macrocode}
\newenvironment{ThesisAbstract}{%
  \begin{alwayssingle}
  \ifx\@thesislang\Langcs
   \chapter*{\AbstractTitlecs}
   \else \ifx\@thesislang\Langsk
    \chapter*{\AbstractTitlesk}
    \else \ifx\@thesislang\Langen
     \chapter*{\AbstractTitleen}
     \else \chapter*{\AbstractTitlecs}
    \fi
   \fi
  \fi}
  {\par\vfil\null
  \end{alwayssingle}
  \newpage}
%    \end{macrocode}
%
% \DescribeMacro\ThesisAbstracten
% \textit{Abstract\/} diplomové práce v~angličtině je 
% možno vysadit pomocí prostředí \texttt{ThesisAbstracten}. 
% Abstract by měl zabírat prostor nejvýše jedné strany. 
%    \begin{macrocode}
\newenvironment{ThesisAbstracten}{%
  \begin{alwayssingle}
  \chapter*{\AbstractTitleen}
  }
  {\par\vfil\null
  \end{alwayssingle}
  \newpage}
%    \end{macrocode}
%
% \DescribeMacro\ThesisKeyWords
% \textit{Klíčová slova\/} oddělená čárkami se vepíší 
% do prostředí \texttt{ThesisKeyWords}. 
%    \begin{macrocode}
\newenvironment{ThesisKeyWords}{%
  \begin{alwayssingle}
  \ifx\@thesislang\Langcs
   \chapter*{\KeyWordsTitlecs}
   \else \ifx\@thesislang\Langsk
    \chapter*{\KeyWordsTitlesk}
    \else \ifx\@thesislang\Langen
     \chapter*{\KeyWordsTitleen}
     \else \chapter*{\KeyWordsTitlecs}
    \fi
   \fi
  \fi}
  {\par\vfill
  \end{alwayssingle}
  \newpage}
%    \end{macrocode}
%
% \DescribeMacro\MainMatter
% Makro \prikaz{MainMatter} nastaví kromě arabského číslování stránek  
% také implicitní styl stránky pro sazbu následujících kapitol. V~tomto
% stylu se do hlavičky stránky vkládá název aktuální kapitoly a od
% ostatního textu se záhlaví oddělí horizontální čarou.
%    \begin{macrocode}
\def\MainMatter{%
  \if@twoside
   \clearpage
   \thispagestyle{empty}
   \cleardoublepage
  \else
   \clearpage
  \fi
  \setcounter{page}{1}
  \pagenumbering{arabic}
  \pagestyle{thesisheadings}
  \parindent 1.5em\relax}
%    \end{macrocode}
%
% Protože je použito dvojí číslování je nutné zadat hyperrefu
% parametry, které zajistí správné odkazování unitř dokumentu.
% \prikaz{usepackage[plainpages=false, pdfpagelabels]\{hyperref\}}
%
% Další text diplomové práce (obsah, úvod, jednotlivé kapitoly a části,
% popřípadě závěr, literatura či dodatky) se již sází standardními
% příkazy. Následuje zjednodušený ukázkový příklad 
% \textit{kostry} diplomové práce.
% \begin{verbatim}
%
% \documentclass[12pt,draft,oneside]{fithesis2}
% \usepackage[utf8]{inputenc} % change when input is not in UTF8 encoding
% \usepackage[plainpages=false, pdfpagelabels]{hyperref}
%
% \thesistitle{Tvorba dokumentu v XML}
% \thesissubtitle{Bakalářská práce}
% \thesisstudent{Jméno Příjmení}
% \thesiswoman{false}
% \thesisfaculty{fi}
% \thesisyear{jaro 2015}
% \thesisadvisor{Jméno Příjmení}
%
% \begin{document}
% \FrontMatter
% \ThesisTitlePage
% 
% \begin{ThesisDeclaration}
% \DeclarationText
% \AdvisorName
% \end{ThesisDeclaration}
%
% \begin{ThesisThanks}
% Zde bude uvedeno \uv{poděkování} ...
% \end{ThesisThanks}
% 
% Obdobně jako poděkování se mohou vysadit shrnutí a klíčová
% slova pomocí prostředí "ThesisAbstract" a "ThesisKeyWords".
%
% \tableofcontents
% \MainMatter
% \chapter*{Úvod}
% Text \ldots
%
% % Následují další kapitoly a podkapitoly, popřípadě závěr, dodatky,
% % seznam literatury či použitých obrázků nebo tabulek, rejstřík
% % a přílohy.
%
% \bibliographystyle{plain}  % bibliografický styl
% \bibliography{mujbisoubor} % soubor s citovanými 
%                            % položkami bibliografie
% \end{document}
% \end{verbatim}
%
% \iffalse
%    \begin{macrocode}

\renewcommand*\l@part[2]{%
  \ifnum \c@tocdepth >-2\relax
    \addpenalty{-\@highpenalty}%
    \addvspace{0.5em \@plus\p@}%
    \begingroup
      \setlength\@tempdima{3em}%
      \parindent \z@ \rightskip \@pnumwidth
      \parfillskip -\@pnumwidth
      {\leavevmode
       \normalfont \bfseries #1\hfil \hb@xt@\@pnumwidth{\hss #2}}\par
       \nobreak
         \global\@nobreaktrue
         \everypar{\global\@nobreakfalse\everypar{}}%
    \endgroup
    \addvspace{0.2em \@plus\p@}%
  \fi}

\renewcommand*\l@chapter[2]{%
  \ifnum \c@tocdepth >\m@ne
    \addpenalty{-\@highpenalty}%
    \vskip 1.0em \@plus\p@
    \setlength\@tempdima{1.5em}%
    \begingroup
      \parindent \z@ \rightskip \@pnumwidth
      \parfillskip -\@pnumwidth
      \leavevmode \bfseries
      \advance\leftskip\@tempdima
      \hskip -\leftskip
      #1\nobreak\hfil \nobreak\hb@xt@\@pnumwidth{\hss #2}\par
      \penalty\@highpenalty
    \endgroup
  \fi}

\renewcommand*\l@chapter{\@dottedtocline{1}{0em}{1.5em}}
\renewcommand*\l@section{\@dottedtocline{2}{1.5em}{2.3em}}
\renewcommand*\l@subsection{\@dottedtocline{2}{3.8em}{3.2em}}
\renewcommand*\l@subsubsection{\@dottedtocline{2}{7.0em}{3.8em}}

%</class>
%
%<*opt>
%<*10pt>
\ProvidesFile{fit10.clo}[2015/03/04 fithesis 0.2.18 (size option)]

\renewcommand{\normalsize}{\fontsize\@xpt{12}\selectfont%
\abovedisplayskip 10\p@ plus2\p@ minus5\p@
\belowdisplayskip \abovedisplayskip
\abovedisplayshortskip  \z@ plus3\p@
\belowdisplayshortskip  6\p@ plus3\p@ minus3\p@
\let\@listi\@listI}

\renewcommand{\small}{\fontsize\@ixpt{11}\selectfont%
\abovedisplayskip 8.5\p@ plus3\p@ minus4\p@
\belowdisplayskip \abovedisplayskip
\abovedisplayshortskip \z@ plus2\p@
\belowdisplayshortskip 4\p@ plus2\p@ minus2\p@
\def\@listi{\leftmargin\leftmargini
\topsep 4\p@ plus2\p@ minus2\p@\parsep 2\p@ plus\p@ minus\p@
\itemsep \parsep}}

\renewcommand{\footnotesize}{\fontsize\@viiipt{9.5}\selectfont%
\abovedisplayskip 6\p@ plus2\p@ minus4\p@
\belowdisplayskip \abovedisplayskip
\abovedisplayshortskip \z@ plus\p@
\belowdisplayshortskip 3\p@ plus\p@ minus2\p@
\def\@listi{\leftmargin\leftmargini %% Added 22 Dec 87
\topsep 3\p@ plus\p@ minus\p@\parsep 2\p@ plus\p@ minus\p@
\itemsep \parsep}}

\renewcommand{\scriptsize}{\fontsize\@viipt{8pt}\selectfont}
\renewcommand{\tiny}{\fontsize\@vpt{6pt}\selectfont}
\renewcommand{\large}{\fontsize\@xiipt{14pt}\selectfont}
\renewcommand{\Large}{\fontsize\@xivpt{18pt}\selectfont}
\renewcommand{\LARGE}{\fontsize\@xviipt{22pt}\selectfont}
\renewcommand{\huge}{\fontsize\@xxpt{25pt}\selectfont}
\renewcommand{\Huge}{\fontsize\@xxvpt{30pt}\selectfont}

%</10pt>
%
%<*11pt>
\ProvidesFile{fit11.clo}[2015/03/04 fithesis 0.2.18 (size option)]

\renewcommand{\normalsize}{\fontsize\@xipt{14}\selectfont%
\abovedisplayskip 11\p@ plus3\p@ minus6\p@
\belowdisplayskip \abovedisplayskip
\belowdisplayshortskip  6.5\p@ plus3.5\p@ minus3\p@
%\abovedisplayshortskip  \z@ plus3\@p
\let\@listi\@listI}

\renewcommand{\small}{\fontsize\@xpt{12}\selectfont%
\abovedisplayskip 10\p@ plus2\p@ minus5\p@ 
\belowdisplayskip \abovedisplayskip
\abovedisplayshortskip  \z@ plus3\p@
\belowdisplayshortskip  6\p@ plus3\p@ minus3\p@
\def\@listi{\leftmargin\leftmargini
\topsep 6\p@ plus2\p@ minus2\p@\parsep 3\p@ plus2\p@ minus\p@
\itemsep \parsep}}

\renewcommand{\footnotesize}{\fontsize\@ixpt{11}\selectfont%
\abovedisplayskip 8\p@ plus2\p@ minus4\p@
\belowdisplayskip \abovedisplayskip
\abovedisplayshortskip \z@ plus\p@ 
\belowdisplayshortskip 4\p@ plus2\p@ minus2\p@
\def\@listi{\leftmargin\leftmargini
\topsep 4\p@ plus2\p@ minus2\p@\parsep 2\p@ plus\p@ minus\p@
\itemsep \parsep}}

\renewcommand{\scriptsize}{\fontsize\@viiipt{9.5pt}\selectfont}
\renewcommand{\tiny}{\fontsize\@vipt{7pt}\selectfont}
\renewcommand{\large}{\fontsize\@xiipt{14pt}\selectfont}
\renewcommand{\Large}{\fontsize\@xivpt{18pt}\selectfont}
\renewcommand{\LARGE}{\fontsize\@xviipt{22pt}\selectfont}
\renewcommand{\huge}{\fontsize\@xxpt{25pt}\selectfont}
\renewcommand{\Huge}{\fontsize\@xxvpt{30pt}\selectfont}

%</11pt>
%
%<*12pt>
\ProvidesFile{fit12.clo}[2015/03/04 fithesis 0.2.18 (size option)]

\renewcommand{\normalsize}{\fontsize\@xiipt{14.5}\selectfont%
\abovedisplayskip 12\p@ plus3\p@ minus7\p@
\belowdisplayskip \abovedisplayskip
\abovedisplayshortskip  \z@ plus3\p@
\belowdisplayshortskip  6.5\p@ plus3.5\p@ minus3\p@
\let\@listi\@listI}

\renewcommand{\small}{\fontsize\@xipt{13.6}\selectfont%
\abovedisplayskip 11\p@ plus3\p@ minus6\p@
\belowdisplayskip \abovedisplayskip
\abovedisplayshortskip  \z@ plus3\p@
\belowdisplayshortskip  6.5\p@ plus3.5\p@ minus3\p@
\def\@listi{\leftmargin\leftmargini %% Added 22 Dec 87
\parsep 4.5\p@ plus2\p@ minus\p@
            \itemsep \parsep
            \topsep 9\p@ plus3\p@ minus5\p@}}

\renewcommand{\footnotesize}{\fontsize\@xpt{12}\selectfont%
\abovedisplayskip 10\p@ plus2\p@ minus5\p@
\belowdisplayskip \abovedisplayskip
\abovedisplayshortskip  \z@ plus3\p@
\belowdisplayshortskip  6\p@ plus3\p@ minus3\p@
\def\@listi{\leftmargin\leftmargini %% Added 22 Dec 87
\topsep 6\p@ plus2\p@ minus2\p@\parsep 3\p@ plus2\p@ minus\p@
\itemsep \parsep}}
            
\renewcommand{\scriptsize}{\fontsize\@viiipt{9.5pt}\selectfont}
\renewcommand{\tiny}{\fontsize\@vipt{7pt}\selectfont}
\renewcommand{\large}{\fontsize\@xivpt{18pt}\selectfont}
\renewcommand{\Large}{\fontsize\@xviipt{22pt}\selectfont}
\renewcommand{\LARGE}{\fontsize\@xxpt{25pt}\selectfont}
\renewcommand{\huge}{\fontsize\@xxvpt{30pt}\selectfont}
\renewcommand{\Huge}{\fontsize\@xxvpt{30pt}\selectfont}

%</12pt>
\let\@normalsize\normalsize
\normalsize

\if@twoside               
   \oddsidemargin 0.75in  
   \evensidemargin 0.4in  
   \marginparwidth 0pt    
\else                     
   \oddsidemargin 0.75in  
   \evensidemargin 0.75in
   \marginparwidth 0pt
\fi
\marginparsep 10pt        

\topmargin 0.4in          
                          
\headheight 20pt          
\headsep 10pt             
\topskip 10pt    
\footskip 30pt 

%<*10pt>
\textheight = 43\baselineskip
\advance\textheight by \topskip
\textwidth 5.0truein
\columnsep 10pt       
\columnseprule 0pt

\footnotesep 6.65pt
\skip\footins 9pt plus 4pt minus 2pt
\floatsep 12pt plus 2pt minus 2pt
\textfloatsep 20pt plus 2pt minus 4pt
\intextsep 12pt plus 2pt minus 2pt
\dblfloatsep 12pt plus 2pt minus 2pt
\dbltextfloatsep 20pt plus 2pt minus 4pt

\@fptop 0pt plus 1fil
\@fpsep 8pt plus 2fil
\@fpbot 0pt plus 1fil
\@dblfptop 0pt plus 1fil
\@dblfpsep 8pt plus 2fil
\@dblfpbot 0pt plus 1fil
\marginparpush 5pt

\parskip 0pt plus 1pt
\partopsep 2pt plus 1pt minus 1pt

%</10pt>
%
%<*11pt>
\textheight = 39\baselineskip
\advance\textheight by \topskip
\textwidth 5.0truein
\columnsep 10pt
\columnseprule 0pt

\footnotesep 7.7pt
\skip\footins 10pt plus 4pt minus 2pt
\floatsep 12pt plus 2pt minus 2pt
\textfloatsep 20pt plus 2pt minus 4pt
\intextsep 12pt plus 2pt minus 2pt
\dblfloatsep 12pt plus 2pt minus 2pt
\dbltextfloatsep 20pt plus 2pt minus 4pt

\@fptop 0pt plus 1fil
\@fpsep 8pt plus 2fil
\@fpbot 0pt plus 1fil
\@dblfptop 0pt plus 1fil
\@dblfpsep 8pt plus 2fil
\@dblfpbot 0pt plus 1fil
\marginparpush 5pt 

\parskip 0pt plus 0pt
\partopsep 3pt plus 1pt minus 2pt

%</11pt>
%
%<*12pt>
\textheight = 37\baselineskip
\advance\textheight by \topskip
\textwidth 5.0truein
\columnsep 10pt
\columnseprule 0pt

\footnotesep 8.4pt
\skip\footins 10.8pt plus 4pt minus 2pt
\floatsep 14pt plus 2pt minus 4pt 
\textfloatsep 20pt plus 2pt minus 4pt
\intextsep 14pt plus 4pt minus 4pt
\dblfloatsep 14pt plus 2pt minus 4pt
\dbltextfloatsep 20pt plus 2pt minus 4pt

\@fptop 0pt plus 1fil
\@fpsep 10pt plus 2fil
\@fpbot 0pt plus 1fil
\@dblfptop 0pt plus 1fil
\@dblfpsep 10pt plus 2fil
\@dblfpbot 0pt plus 1fil
\marginparpush 7pt

\parskip 0pt plus 0pt
\partopsep 3pt plus 2pt minus 2pt

%</12pt>
\@lowpenalty   51
\@medpenalty  151
\@highpenalty 301
\@beginparpenalty -\@lowpenalty
\@endparpenalty   -\@lowpenalty
\@itempenalty     -\@lowpenalty

\let\@fichapters\False
\ifx\clsclass\Scrreprtcls\let\@fichapters\True\fi
\ifx\clsclass\RapportIcls\let\@fichapters\True\fi
\ifx\clsclass\RapportIIIcls\let\@fichapters\True\fi
\ifx\@fichapters\True
  \def\@makechapterhead#1{%
    {%
      \setlength\parindent{\z@}%
      \setlength\parskip  {\z@}%
      \ifnum
        \c@secnumdepth >\m@ne
	      \par\nobreak
	      \vskip 10\p@
      \fi
      \Large \ChapFont \thechapter{} \space #1\par
      \nobreak
      \vskip 20\p@
    }%
  }

  \def\@makeschapterhead#1{%
    {%
      \setlength\parindent{\z@}%
      \setlength\parskip  {\z@}%
      \Large \ChapFont #1\par
      \nobreak
      \vskip 20\p@
    }%
  }

  \def\chapter{%
     \clearpage
     \thispagestyle{plain}
     \global\@topnum\z@ 
     \@afterindentfalse  
     \secdef\@chapter\@schapter
   }

  \def\@chapter[#1]#2{%
    \ifnum \c@secnumdepth
      >\m@ne
	    \refstepcounter{chapter}%
	    \typeout{\@chapapp\space\thechapter.}% 
	    \addcontentsline{toc}{chapter}{\protect
	    \numberline{\thechapter}\bfseries #1}
    \else%
    	\addcontentsline{toc}{chapter}{\bfseries #1}
    \fi
    \chaptermark{#1}%
    \addtocontents{lof}%
	  {\protect\addvspace{4\p@}} 
    \addtocontents{lot}%
	  {\protect\addvspace{4\p@}} 
    \if@twocolumn                   
	    \@topnewpage[\@makechapterhead{#2}]%
    \else
      \@makechapterhead{#2}%
	    \@afterheading          
    \fi
  }

  %\def\@schapter#1{\if@twocolumn \@topnewpage[\@makeschapterhead{#1}]%
  %        \else \@makeschapterhead{#1}%
  %              \markright{#1}
  %              \@afterheading\fi}
\fi

\def\section{\@startsection {section}{1}{\z@}{-3.5ex plus-1ex minus
    -.2ex}{2.3ex plus.2ex}{\reset@font\large\bfseries}}
\def\subsection{\@startsection{subsection}{2}{\z@}{-3.25ex plus-1ex
    minus-.2ex}{1.5ex plus.2ex}{\reset@font\normalsize\bfseries}}
\def\subsubsection{\@startsection{subsubsection}{3}{\z@}{-3.25ex plus   
    -1ex minus-.2ex}{1.5ex plus.2ex}{\reset@font\normalsize}}
\def\paragraph{\@startsection
    {paragraph}{4}{\z@}{3.25ex plus1ex minus.2ex}{-1em}{\reset@font
    \normalsize\bfseries}}
\def\subparagraph{\@startsection
     {subparagraph}{4}{\parindent}{3.25ex plus1ex minus
     .2ex}{-1em}{\reset@font\normalsize\bfseries}}

\setcounter{secnumdepth}{2}

\def\appendix{\par
  \setcounter{chapter}{0}%
  \setcounter{section}{0}%
  \def\@chapapp{\appendixname}%
  \def\thechapter{\Alph{chapter}}}

\leftmargini 2.5em
\leftmarginii 2.2em     % > \labelsep + width of '(m)'
\leftmarginiii 1.87em   % > \labelsep + width of 'vii.'
\leftmarginiv 1.7em     % > \labelsep + width of 'M.'
\leftmarginv 1em
\leftmarginvi 1em

\leftmargin\leftmargini
\labelsep .5em
\labelwidth\leftmargini\advance\labelwidth-\labelsep

%<*10pt>
\def\@listI{\leftmargin\leftmargini \parsep 4\p@ plus2\p@ minus\p@%
\topsep 8\p@ plus2\p@ minus4\p@
\itemsep 4\p@ plus2\p@ minus\p@}

\let\@listi\@listI
\@listi

\def\@listii{\leftmargin\leftmarginii
   \labelwidth\leftmarginii\advance\labelwidth-\labelsep
   \topsep 4\p@ plus2\p@ minus\p@
   \parsep 2\p@ plus\p@ minus\p@
   \itemsep \parsep}

\def\@listiii{\leftmargin\leftmarginiii
    \labelwidth\leftmarginiii\advance\labelwidth-\labelsep
    \topsep 2\p@ plus\p@ minus\p@
    \parsep \z@ \partopsep\p@ plus\z@ minus\p@
    \itemsep \topsep}

\def\@listiv{\leftmargin\leftmarginiv
     \labelwidth\leftmarginiv\advance\labelwidth-\labelsep}
   
\def\@listv{\leftmargin\leftmarginv
     \labelwidth\leftmarginv\advance\labelwidth-\labelsep}
   
\def\@listvi{\leftmargin\leftmarginvi
     \labelwidth\leftmarginvi\advance\labelwidth-\labelsep}
%</10pt>
%
%<*11pt>
\def\@listI{\leftmargin\leftmargini \parsep 4.5\p@ plus2\p@ minus\p@
\topsep 9\p@ plus3\p@ minus5\p@
\itemsep 4.5\p@ plus2\p@ minus\p@}

\let\@listi\@listI
\@listi

\def\@listii{\leftmargin\leftmarginii
   \labelwidth\leftmarginii\advance\labelwidth-\labelsep
   \topsep 4.5\p@ plus2\p@ minus\p@
   \parsep 2\p@ plus\p@ minus\p@
   \itemsep \parsep}

\def\@listiii{\leftmargin\leftmarginiii
    \labelwidth\leftmarginiii\advance\labelwidth-\labelsep
    \topsep 2\p@ plus\p@ minus\p@
    \parsep \z@ \partopsep \p@ plus\z@ minus\p@
    \itemsep \topsep}

\def\@listiv{\leftmargin\leftmarginiv
     \labelwidth\leftmarginiv\advance\labelwidth-\labelsep}
   
\def\@listv{\leftmargin\leftmarginv
     \labelwidth\leftmarginv\advance\labelwidth-\labelsep}
    
\def\@listvi{\leftmargin\leftmarginvi
     \labelwidth\leftmarginvi\advance\labelwidth-\labelsep}
%</11pt>
%
%<*12pt>
\def\@listI{\leftmargin\leftmargini \parsep 5\p@ plus2.5\p@ minus\p@
\topsep 10\p@ plus4\p@ minus6\p@
\itemsep 5\p@ plus2.5\p@ minus\p@}

\let\@listi\@listI
\@listi

\def\@listii{\leftmargin\leftmarginii
   \labelwidth\leftmarginii\advance\labelwidth-\labelsep
   \topsep 5\p@ plus2.5\p@ minus\p@
   \parsep 2.5\p@ plus\p@ minus\p@
   \itemsep \parsep}

\def\@listiii{\leftmargin\leftmarginiii
    \labelwidth\leftmarginiii\advance\labelwidth-\labelsep
    \topsep 2.5\p@ plus\p@ minus\p@
    \parsep \z@ \partopsep \p@ plus\z@ minus\p@
    \itemsep \topsep}

\def\@listiv{\leftmargin\leftmarginiv
     \labelwidth\leftmarginiv\advance\labelwidth-\labelsep}
   
\def\@listv{\leftmargin\leftmarginv
     \labelwidth\leftmarginv\advance\labelwidth-\labelsep}
    
\def\@listvi{\leftmargin\leftmarginvi
     \labelwidth\leftmarginvi\advance\labelwidth-\labelsep}
%</12pt>
%</opt>
%    \end{macrocode}
% \fi
%
% 
% \iffalse
%    \begin{macrocode}
%<*oldclass>
\NeedsTeXFormat{LaTeX2e}
\ProvidesClass{oldfithesis}[2015/03/04 old fithesis will load newer fithesis2 MU thesis class]

\errmessage{%
  You are using the fithesis class, which has been deprecated.
  The fithesis2 class will be used instead.
  For more information, see <https://www.fi.muni.cz/tech/unix/tex/fithesis.xhtml>%
}

\ifx\clsclass\undefined
 \def\clsclass{fithesis2}
\fi
\LoadClass{\clsclass}
\endinput
%</oldclass>
%    \end{macrocode}
% \fi
%
